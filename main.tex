\pdfminorversion=4
\documentclass[aspectratio=169]{beamer}

\mode<presentation>
{
  \usetheme{default}
  \usecolortheme{default}
  \usefonttheme{default}
  \setbeamertemplate{navigation symbols}{}
  \setbeamertemplate{caption}[numbered]
  \setbeamertemplate{footline}[frame number]  % or "page number"
  \setbeamercolor{frametitle}{fg=white}
  \setbeamercolor{footline}{fg=black}
}

\usepackage[english]{babel}
\usepackage[utf8x]{inputenc}
\usepackage{tikz}
\usepackage{courier}
\usepackage{array}
\usepackage{bold-extra}
\usepackage{minted}
\usepackage[thicklines]{cancel}
\usepackage{fancyvrb}

\xdefinecolor{dianablue}{rgb}{0.18,0.24,0.31}
\xdefinecolor{darkblue}{rgb}{0.1,0.1,0.7}
\xdefinecolor{darkgreen}{rgb}{0,0.5,0}
\xdefinecolor{darkgrey}{rgb}{0.35,0.35,0.35}
\xdefinecolor{darkorange}{rgb}{0.8,0.5,0}
\xdefinecolor{darkred}{rgb}{0.7,0,0}
\definecolor{darkgreen}{rgb}{0,0.6,0}
\definecolor{mauve}{rgb}{0.58,0,0.82}

\title[2019-07-10-awkward-plans]{Awkward plans}
\author{Jim Pivarski}
\institute{Princeton University -- IRIS-HEP}
\date{July 10, 2019}

\usetikzlibrary{shapes.callouts}

\begin{document}

\logo{\pgfputat{\pgfxy(0.11, 7.4)}{\pgfbox[right,base]{\tikz{\filldraw[fill=dianablue, draw=none] (0 cm, 0 cm) rectangle (50 cm, 1 cm);}\mbox{\hspace{-8 cm}\includegraphics[height=1 cm]{princeton-logo-long.png}\hspace{0.1 cm}\raisebox{0.1 cm}{\includegraphics[height=0.8 cm]{iris-hep-logo-long.png}}\hspace{0.1 cm}}}}}

\begin{frame}
  \titlepage
\end{frame}

\logo{\pgfputat{\pgfxy(0.11, 7.4)}{\pgfbox[right,base]{\tikz{\filldraw[fill=dianablue, draw=none] (0 cm, 0 cm) rectangle (50 cm, 1 cm);}\mbox{\hspace{-8 cm}\includegraphics[height=1 cm]{princeton-logo.png}\hspace{0.1 cm}\raisebox{0.1 cm}{\includegraphics[height=0.8 cm]{iris-hep-logo.png}}\hspace{0.1 cm}}}}}

% Uncomment these lines for an automatically generated outline.
%\begin{frame}{Outline}
%  \tableofcontents
%\end{frame}

% START START START START START START START START START START START START START

\begin{frame}{Context}
\large
\vspace{0.35 cm}
Awkward-array is the end of a long chain of ideas about columnar analysis:

\vspace{0.25 cm}
\begin{enumerate}
  \item \textcolor{darkblue}{Femtocode} was originally envisioned as a combined language, columnar processor, and distributed system: a full-package service for physics.
  \item To be more realistic, I separated the language and columnar processing from the distributed processing, leaving the latter to others.
  \item \textcolor{darkblue}{Shredtypes} $\to$ \textcolor{darkblue}{Quiver} (for Arrow) $\to$ \textcolor{darkblue}{OAMap} were iterations on expressing an abstract type system with an internal columnar representation.
  \item \textcolor{darkblue}{Uproot 2} had a ``minimal columnar data type'': a {\tt JaggedArray} class.
  \item Users found {\tt JaggedArrays} useful despite the fact that it doesn't hide the columnar representation (this was a surprise to me).
  \item \textcolor{darkblue}{Awkward-array} built the abstract type system to columnar representation in the other direction: bottom-up, from physical arrays to abstract types.
  \item \textcolor{darkblue}{Awkward-array} is the first in this series to be widely used; I'm getting a lot of feedback from Coffea and \textcolor{darkblue}{uproot 3} users.
\end{enumerate}
\end{frame}

\begin{frame}{Up to now: one specification, many implementations}
\large
\vspace{0.5 cm}

Four of my ``ongoing projects'' are different implementations of awkward, all to adhere to the {\tt specification.adoc} (which is now out of date).

\includegraphics[width=\linewidth]{projects.pdf}
\end{frame}

\begin{frame}{Just finished a homogenization/documentation campaign}
\vspace{0.25 cm}
\begin{columns}
\column{0.35\linewidth}
\includegraphics[width=\linewidth]{awkward-documentation.png}

\column{0.55\linewidth}
``Homogenization'' because a common user complaint is that methods defined on one class (e.g. {\tt JaggedArray}) don't work on another class (e.g. {\tt MaskedArray} of {\tt JaggedArray}).

\vspace{0.5 cm}
While documenting (100 pages of examples), I ensured that a suite of ``high-level'' methods does something meaningful on every array class.
\end{columns}
\end{frame}

\begin{frame}{Problems}
\large
\vspace{0.5 cm}
\begin{enumerate}\setlength{\itemsep}{0.2 cm}
\item Writing the same method on 14 classes $\times$ 4 implementations is a problem for developer FTEs now and maintainance later.

\textcolor{gray}{(Remember, I haven't touched awkward-numba since February!)}

\item Users need a better separation between high-level and low-level.

\begin{itemize}\normalsize\setlength{\itemsep}{0.2 cm}
\item Structural classes, such as {\tt ChunkedArray} of {\tt VirtualArrays} of {\tt X} to make array {\tt X} lazy-loading, should be hidden.

\item Named methods, like \mintinline{python}{x.cross(y)} meaning ``cross-join,'' compete for the same namespace with column names and domain-specific methods, like \mintinline{python}{x.cross(y)} meaning ``3-D cross product.''
\end{itemize}

\item Too many features are opt-in. Numba integration will fail and Pandas integration will have a subtle performance bug if the user doesn't \mintinline{python}{import awkward.numba} and \mintinline{python}{import awkward.pandas} first.

\end{enumerate}
\end{frame}

\begin{frame}{Solution to \#1: implement at most twice}
\vspace{0.1 cm}

\begin{columns}
\column{0.55\linewidth}
\includegraphics[width=\linewidth]{awkward-1-0-layers.pdf}

\column{0.5\linewidth}
Put all CPU implementations behind an \mintinline{c++}{extern "C"} interface (stateless).

\begin{itemize}
\item C++ classes (e.g. {\tt JaggedArray}) provide structure and ownership rules.
\item Numba (e.g {\tt JaggedArrayModel}) extensions associate Python classes in \mintinline{python}{@numba.jit} functions with the C implementations.
\item Structure classes mirrored between C++ and Python.
\item Structure classes hidden inside {\tt Array}.
\end{itemize}

Maybe later write \mintinline{c++}{extern "C"} functions for GPU as well.
\end{columns}
\end{frame}



\end{document}
